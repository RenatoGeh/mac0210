\documentclass[12pt]{article}

\usepackage[brazilian]{babel}
\usepackage[utf8]{inputenc}
\usepackage{graphicx}
\usepackage{mathtools}
\usepackage{amsthm}
\usepackage{amssymb}
\usepackage{thmtools,thm-restate}
\usepackage{amsfonts}
\usepackage{hyperref}
\usepackage[singlelinecheck=false]{caption}
\usepackage[backend=biber,url=true,doi=true,eprint=false,style=numeric]{biblatex}
\usepackage{enumitem}
\usepackage[justification=centering]{caption}
\usepackage{indentfirst}
\usepackage{algorithm}
\usepackage[noend]{algpseudocode}
\usepackage{listings}
\usepackage[x11names,rgb,table]{xcolor}
\usepackage{tikz}
\usepackage{hyperref}
\usepackage{subcaption}
\usepackage{booktabs}
\usepackage{linegoal}
\usepackage{geometry}
\usetikzlibrary{snakes,arrows,shapes}

\addbibresource{references.bib}
\graphicspath{{imgs/}}

\makeatletter
\def\subsection{\@startsection{subsection}{3}%
  \z@{.5\linespacing\@plus.7\linespacing}{.1\linespacing}%
  {\normalfont}}
\makeatother

\makeatletter
\patchcmd{\@setauthors}{\MakeUppercase}{}{}{}
\makeatother

\DeclareMathOperator*{\argmin}{arg\,min}
\DeclareMathOperator*{\argmax}{arg\,max}
\DeclareMathOperator*{\Val}{\text{Val}}
\DeclareMathOperator*{\Ch}{\text{Ch}}
\DeclareMathOperator*{\Pa}{\text{Pa}}
\DeclareMathOperator*{\Sc}{\text{Sc}}
\newcommand{\ov}{\overline}
\newcommand{\tsup}{\textsuperscript}

\newcommand\defeq{\mathrel{\overset{\makebox[0pt]{\mbox{\normalfont\tiny\sffamily def}}}{=}}}

\newcommand{\algorithmautorefname}{Algorithm}
\algrenewcommand\algorithmicrequire{\textbf{Entrada}}
\algrenewcommand\algorithmicensure{\textbf{Saída}}
\algrenewcommand\algorithmicif{\textbf{se}}
\algrenewcommand\algorithmicthen{\textbf{então}}
\algrenewcommand\algorithmicelse{\textbf{senão}}
\algrenewcommand\algorithmicfor{\textbf{para todo}}
\algrenewcommand\algorithmicdo{\textbf{faça}}
\algnewcommand{\LineComment}[1]{\State\,\(\triangleright\) #1}

\captionsetup[table]{labelsep=space}

\theoremstyle{plain}

\newcounter{dummy-def}\numberwithin{dummy-def}{section}
\newtheorem{definition}[dummy-def]{Definition}
\newcounter{dummy-thm}\numberwithin{dummy-thm}{section}
\newtheorem{theorem}[dummy-thm]{Theorem}
\newcounter{dummy-prop}\numberwithin{dummy-prop}{section}
\newtheorem{proposition}[dummy-prop]{Proposition}
\newcounter{dummy-corollary}\numberwithin{dummy-corollary}{section}
\newtheorem{corollary}[dummy-corollary]{Corollary}
\newcounter{dummy-lemma}\numberwithin{dummy-lemma}{section}
\newtheorem{lemma}[dummy-lemma]{Lemma}
\newcounter{dummy-ex}\numberwithin{dummy-ex}{section}
\newtheorem{exercise}[dummy-ex]{Exercise}
\newcounter{dummy-eg}\numberwithin{dummy-eg}{section}
\newtheorem{example}[dummy-eg]{Example}

\numberwithin{equation}{section}

\newcommand{\set}[1]{\mathbf{#1}}
\newcommand{\pr}{\text{P}}
\newcommand{\eps}{\varepsilon}
\newcommand{\ddspn}[2]{\frac{\partial#1}{\partial#2}}
\newcommand{\iddspn}[2]{\partial#1/\partial#2}
\newcommand{\indep}{\perp}
\renewcommand{\implies}{\Rightarrow}

\newcommand{\bigo}{\mathcal{O}}

\setlength{\parskip}{1em}

\lstset{frameround=fttt,
	numbers=left,
	breaklines=true,
	keywordstyle=\bfseries,
	basicstyle=\ttfamily,
}

\newcommand{\code}[1]{\lstinline[mathescape=true]{#1}}
\newcommand{\mcode}[1]{\lstinline[mathescape]!#1!}

\title{%
  Curvas de Bézier\\~\\
  {\normalfont EP1 de MAC0210}
}
\author{Renato Lui Geh, NUSP: 8536030}
\date{}

\begin{document}

\maketitle

\section{Introdução}

O EP foi feito na linguagem Python. Foi usada a biblioteca pyglet\footnote{Disponível em
\url{https://bitbucket.org/pyglet/pyglet/wiki/Home}}, que age como um \textit{wrapper} de OpenGL
para Python. Toda a parte de desenho e GUI foi feita com esta biblioteca.

Para rodar o EP, é preciso do Python 3 e que a biblioteca pyglet esteja instalada. Além disso,
alguns cálculos foram feitos com NumPy.

Para executar o EP, faça:

\begin{lstlisting}[numbers=none]
  python3 main.py
\end{lstlisting}

\section{Uso}

O arquivo \code{main.py} roda o programa, que constrói uma janela com um canvas vazio. A janela
inicial apresenta as instruções de uso e algumas informações do ambiente atual.

\begin{figure}[h]
  \centering\frame{\includegraphics[width=0.8\textwidth]{imgs/initial.png}}
\end{figure}

No canto inferior esquerdo ficam as informações atuais do ambiente. No canto inferior direito é
possível encontrar os comandos disponíveis.

Com o botão direito do mouse, cria-se um dos pontos da curva de Bézier. Se o grau da curva é dois,
são precisos três pontos; se três, quatro pontos são necessários.

\begin{figure}[h]
  \centering\frame{\includegraphics[width=0.8\textwidth]{imgs/1.png}}
\end{figure}

Após criadas, é possível selecionar uma curva com o botão esquerdo. Ao clicar com o botão esquerdo,
seleciona-se a curva mais próxima do cursor. O modo de debug de distância (hotkey D) desenha o
vetor diferença entre a posição do cursor e o ponto mais próximo da curva ao cursor. No painel de
informação também é possível ver a distância calculada ``Distance to curve''.

\begin{figure}[h]
  \centering\frame{\includegraphics[width=0.8\textwidth]{imgs/2.png}}
\end{figure}

Após selecionada, a curva pode ser alterada modificando as posições dos seus pontos. Selecionam-se
os pontos clicando nas bolas rosas da curva. O vértice selecionado é então redesenhado com a cor
verde. Alternativamente, é possível ciclar pelos pontos de forma mais rápida usando TAB. Após
selecionado, o ponto pode ser movido apertando com o botão esquerdo e arrastando.

\begin{figure}[h]
  \centering\frame{\includegraphics[width=0.8\textwidth]{imgs/3.png}}
\end{figure}

Com a tecla G, é possível mudar o grau das curvas a serem criadas. Foram implementadas curvas de
grau dois e três.

\begin{figure}[h]
  \centering\frame{\includegraphics[width=0.8\textwidth]{imgs/4.png}}
\end{figure}

O programa lida bem com várias curvas, e mesmo quando há interseção das curvas e de seus vértices,
é possível seleciona-las e movimenta-las livremente.

\begin{figure}[h]
  \centering\frame{\includegraphics[width=0.8\textwidth]{imgs/5.png}}
\end{figure}

É possível entrar em modo apresentação (hotkey P). Quando em modo de apresentação, o programa deixa
de desenhar os pontos da curva, os paineis de instrução e comandos, e deixa de colorir as retas
selecionadas.

\begin{figure}[h]
  \centering\frame{\includegraphics[width=0.8\textwidth]{imgs/6.png}}
\end{figure}

É possível remover uma curva selecionando-a e apertando a tecla DELETE. Para limpar o canvas e
remover todas as curvas, aperte SHIFT+X.

\section{Estrutura do código}

O arquivo \code{main.py} que é usado para começar o EP cria um objeto da classe \code{Frame},
localizado em \code{frame.py}. Esta classe cuida da criação da janela, do desenho de cada elemento
do canvas, e do input de mouse e teclado.

Toda vez que o botão direito é usado dentro do canvas, o \code{Frame} registra os pontos e verifica
o número de pontos criados. Se este número for igual ao grau da curva de Bézier somado a um, então
cria-se um novo objeto \code{Bezier} (em \code{bezier.py}), que representa a curva.

Esta classe trata de tudo sobre a curva. Seu construtor toma os pontos da curva e seu grau. Deste
jeito podemos computar os coeficientes do polinômio (representado pela variável \code{Bezier.k}) e
os coeficientes de sua derivada (variável \code{Bezier.dk}). Esse calculo é feito pelo método
\code{recompute}, que é chamado toda vez que mudamos os pontos da curva.

A curva é desenhada pelo método de DeCasteljau. A função \code{_de_casteljau} e
\code{_de_casteljau2} desenham as curvas de grau três e dois respectivamente.

Para computar a distância de uma curva de Bézier $B$ até um ponto $q$, usa-se o método
\code{Bezier.distance}. A distância é computada tomando o mínimo da distância euclideana:

\begin{equation*}
  \argmin_{0\leq t\leq 1} d(q, B) = \sqrt{\left(B_x(t)-q_x\right)^2+\left(B_y(t)-q_y\right)^2}
\end{equation*}

Derivando e igualando a zero, temos um polinômio cujas raízes representam os máximos e mínimos da
distância. Usamos a função \code{np.roots} da biblioteca NumPy para achar as raízes do polinômio.
Em seguida, descartamos todas as raízes que não estejam em $\mathbb{R}$ e tomamos o mínimo dos
restantes.

Os métodos \code{Bezier.f} e \code{Bezier.df} tomam um $t$ e avaliam $(B_x(t), B_y(t))$ e
$\ddspn{}{t}(B_x(t),B_y(t))$.

A classe \code{Label} desenha os painéis de informação e instrução, e pode ser encontrada no
arquivo \code{label.py}.

O arquivo \code{utils.py} contém funções de conveniência, como por exemplo a distância de um ponto
até um segmento de reta, o mínimo entre dois pontos e um outro ponto, e uma rotina para desenhar
bolas através de triângulos (\code{GL_TRIANGLE_FAN} do OpenGL).

\printbibliography[]

\end{document}
